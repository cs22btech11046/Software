

\let\negmedspace\undefined
\let\negthickspace\undefined

\documentclass[journal,12pt,twocolumn]{IEEEtran}

%\documentclass[conference]{IEEEtran}
%\IEEEoverridecommandlockouts
% The preceding line is only needed to identify funding in the first footnote. If that is unneeded, please comment it out.
\usepackage{cite}
\usepackage{amsmath,amssymb,amsfonts,amsthm}
\usepackage{algorithmic}
\usepackage{graphicx}
\usepackage{textcomp}
\usepackage{xcolor}
\usepackage{txfonts}
\usepackage{listings}
\usepackage{enumitem}
\usepackage{mathtools}
\usepackage{gensymb}
\usepackage[breaklinks=true]{hyperref}
\usepackage{tkz-euclide} % loads  TikZ and tkz-base
\usepackage{listings}
\usepackage{graphicx}
\usepackage{caption} 
                                           
\DeclareMathOperator*{\Res}{Res}
%\renewcommand{\baselinestretch}{2}
\renewcommand\thesection{\arabic{section}}
\renewcommand\thesubsection{\thesection.\arabic{subsection}}
\renewcommand\thesubsubsection{\thesubsection.\arabic{subsubsection}}

\renewcommand\thesectiondis{\arabic{section}}
\renewcommand\thesubsectiondis{\thesectiondis.\arabic{subsection}}
\renewcommand\thesubsubsectiondis{\thesubsectiondis.\arabic{subsubsection}}

% correct bad hyphenation here
\hyphenation{op-tical net-works semi-conduc-tor}
\def\inputGnumericTable{}                                 %%

\lstset{
%language=C,
frame=single, 
breaklines=true,
columns=fullflexible
}

\begin{document}
%

\newtheorem{theorem}{Theorem}[section]
\newtheorem{problem}{Problem}
\newtheorem{proposition}{Proposition}[section]
\newtheorem{lemma}{Lemma}[section]
\newtheorem{corollary}[theorem]{Corollary}
\newtheorem{example}{Example}[section]
\newtheorem{definition}[problem]{Definition}

%\newcommand{\BEQA}{\begin{eqnarray}}
%\newcommand{\EEQA}{\end{eqnarray}}
%\newcommand{\define}{\stackrel{\triangle}{=}}

\bibliographystyle{IEEEtran}


\newcommand{\define}{\stackrel{\triangle}{=}}
\newcommand{\permcomb}[4][0mu]{{{}^{#3}\mkern#1#2_{#4}}}
\newcommand{\comb}[1][-1mu]{\permcomb[#1]{C}}

%\bibliographystyle{ieeetr}

\providecommand{\mbf}{\mathbf}
\providecommand{\pr}[1]{\ensuremath{\Pr\left(#1\right)}}
\providecommand{\qfunc}[1]{\ensuremath{Q\left(#1\right)}}
\providecommand{\sbrak}[1]{\ensuremath{{}\left[#1\right]}}
\providecommand{\lsbrak}[1]{\ensuremath{{}\left[#1\right.}}
\providecommand{\rsbrak}[1]{\ensuremath{{}\left.#1\right]}}
\providecommand{\brak}[1]{\ensuremath{\left(#1\right)}}
\providecommand{\lbrak}[1]{\ensuremath{\left(#1\right.}}
\providecommand{\rbrak}[1]{\ensuremath{\left.#1\right)}}
\providecommand{\cbrak}[1]{\ensuremath{\left\{#1\right\}}}
\providecommand{\lcbrak}[1]{\ensuremath{\left\{#1\right.}}
\providecommand{\rcbrak}[1]{\ensuremath{\left.#1\right\}}}
\theoremstyle{remark}
\newtheorem{rem}{Remark}
\newcommand{\sgn}{\mathop{\mathrm{sgn}}}
\providecommand{\abs}[1]{\left\vert#1\right\vert}
\providecommand{\res}[1]{\Res\displaylimits_{#1}} 
\providecommand{\norm}[1]{\left\lVert#1\right\rVert}
%\providecommand{\norm}[1]{\lVert#1\rVert}
\providecommand{\mtx}[1]{\mathbf{#1}}
\providecommand{\mean}[1]{E\left[ #1 \right]}
\providecommand{\fourier}{\overset{\mathcal{F}}{ \rightleftharpoons}}
%\providecommand{\hilbert}{\overset{\mathcal{H}}{ \rightleftharpoons}}
\providecommand{\system}{\overset{\mathcal{H}}{ \longleftrightarrow}}
	%\newcommand{\solution}[2]{\textbf{Solution:}{#1}}
\newcommand{\solution}{\noindent \textbf{Solution: }}
\newcommand{\cosec}{\,\text{cosec}\,}
\providecommand{\dec}[2]{\ensuremath{\overset{#1}{\underset{#2}{\gtrless}}}}
\newcommand{\myvec}[1]{\ensuremath{\begin{pmatrix}#1\end{pmatrix}}}
\newcommand{\mydet}[1]{\ensuremath{\begin{vmatrix}#1\end{vmatrix}}}

\let\vec\mathbf


\vspace{3cm}

\title{
\textbf{Software Asssignment Report}
}

\author{CS22BTECH11046\\P.Yasaswini}	


\maketitle

\newpage

%\tableofcontents

\bigskip

\renewcommand{\thefigure}{\theenumi}
\renewcommand{\thetable}{\theenumi}



\section{Introduction}
 The program written in Python allows users to play a random selection of songs from a specified playlist and provides basic music playback controls using Pygame library.

\section{Code Overview}
The code begins by importing the necessary modules, including \texttt{pygame}, \texttt{random}, and \texttt{os}. It defines a function called \texttt{play\_random\_song} which takes a playlist as input and plays the songs in a random order.

\subsection{Shuffling the Playlist}
 The code first shuffles the playlist using the \texttt{random.shuffle()} function. This rearranges the order of the songs randomly, creating a randomized playlist for playback.
Next, the code initializes the Pygame mixer module using \texttt{pygame.mixer.init()}. This step is necessary to handle music playback.

\subsection{Playing Songs}
The code iterates over each song in the shuffled playlist. For each song, it performs the following steps:
\begin{enumerate}
    \item Prints the current song being played.
    \item Loads the song using \texttt{pygame.mixer.music.load()}.
    \item Plays the loaded song using \texttt{pygame.mixer.music.play()}.
\end{enumerate}
While a song is playing, the program prompts the user to enter a command via the \texttt{input()} function. The available commands are "pause," "unpause," "next," and "exit." The program responds accordingly to each command, pausing, resuming, skipping to the next song, or exiting the program.

\begin{figure}[htbp]
\centering
\includegraphics[scale=0.4]{scn.png}
\captionsetup{justification=centering}
\caption{Output}
\end{figure}

\section{Conclusion}
The provided code presents a functional implementation of a randomized music player using the Pygame library in Python. It allows users to play a random selection of songs from a playlist and provides basic music playback controls. The program can be further extended and customized based on specific requirements or additional features desired.



\end{document}


